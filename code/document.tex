\documentclass[9pt]{amsart}
\usepackage[utf8]{inputenc}
\usepackage{amsmath, amssymb, amsthm}
\usepackage{mathrsfs, graphicx, tikz}
\usepackage[left=3cm, right=3cm, bottom=3.4cm]{geometry}
\usepackage{hyperref}
\usepackage{fancyhdr}

\renewcommand{\qedsymbol}{\emph{(end of proof)}}

\theoremstyle{plain}
\newtheorem{theorem}{Theorem}
\renewcommand{\thetheorem}{\Roman{theorem}}

\newtheorem{definition}{Definition}
\renewcommand{\thedefinition}{\Roman{definition}}

\newtheorem{lemma}{Lemma}
\renewcommand{\thelemma}{\Roman{lemma}}

\newtheorem{conjecture}{Conjecture}
\renewcommand{\theconjecture}{\Roman{conjecture}}


\title{\textbf{Investigating square peg problem by Delta Rotations and surging}}
\author{Jayant Sharma}
\date{\today}

\pagestyle{fancy}
\fancyhf{}
\fancyhead[L]{\emph{Investigating square peg problem by Delta Rotations and surging, Jayant Sharma}}
\fancyhead[R]{\thepage}
\renewcommand{\headrulewidth}{0.4pt}
\renewcommand{\footrulewidth}{0pt}

\begin{document}
\maketitle
\begin{abstract}
In 1911, Otto Toeplitz, a German mathematician, proposed the celebrated conjecture: \emph{Does every simple, closed, non-self-intersecting Jordan curve inscribe a square—that is, can four points on the curve always be found that form the vertices of a square?}

Over the years, numerous partial results have deepened our understanding of this geometric-topological problem, with the conjecture being resolved for many special classes of curves. Yet, the general case for arbitrary Jordan curves remains unsolved.

In this paper, we introduce a novel technique called the \textbf{method of delta-rotations}, which considers a configuration of two perpendicular line segments entirely contained within the interior of a Jordan curve. This configuration is rotated and slid along the curve’s periphery such that the orthogonality is preserved and all four endpoints remain on the curve.

To maintain smooth and continuous mappings—and to avoid obstructions during deformation—we propose an additional technique called the \textbf{method of surging}, which shifts one of the lines parallel to the other, preserving orthogonality while circumventing local blockages.

If, through this process, the perpendicular lines are shown to exchange lengths during their motion, then by continuity and the \emph{Intermediate Value Theorem}, there must exist a configuration where both lines attain equal lengths, thereby forming the diagonals of an inscribed square.
\end{abstract}
\tableofcontents

\nocite{*}

\section{Preliminaries}

We begin by introducing the notion of the $\,^n \Delta_k$ rotation and associated sets.

Let $S \subseteq \mathbb{R}^n$ be a set of points in $n$-dimensional Euclidean space. The $\,^n \Delta_k$ configuration is defined as a system of $k$ lines in $\mathbb{R}^n$ that are mutually equiangular and intersect at a single point $P(n, k)$, referred to as the \textbf{configuration point}.

Corresponding to this configuration, we define the $\,^n \Delta_k$ sets as a collection of $k+1$ independent sets:  
\begin{itemize}
    \item $k$ of these are termed \textit{parametric sets}, each representing the lengths assigned to the respective $k$ lines of the configuration.  
    \item The remaining set is called the \textit{angular set}, which governs the direction or orientation of the entire configuration in space.
\end{itemize}

At each instance of rotation determined by an element of the angular set, the $k$ lines act as position vectors with respective magnitudes from the parametric sets. Together, they describe a subset of points in $\mathbb{R}^n$ relative to the configuration point $P(n, k)$.

In this paper, we primarily consider the specific configuration $\,^2 \Delta_4$, representing two mutually perpendicular lines in the plane intersecting at a point $P(2, 4)$, which we shall simply denote by $P$. For ease of reference, we denote the configuration $\,^2 \Delta_4$ by $\Delta$ throughout the paper.

Now, we formally state the central conjecture under consideration:

\begin{conjecture}[Square Peg Problem]
Every simple, closed, non-self-intersecting Jordan curve $C$ inscribes a square; that is, there exists a square whose four vertices all lie on the boundary of $C$.
\end{conjecture}

\textbf{Conjecture 1}, widely known as the \emph{Square Peg Problem}, was first posed by \emph{Otto Toeplitz} in 1911 and remains one of the most intriguing open problems in elementary geometry.


\bibliographystyle{plain}
\bibliography{document}

\end{document}
