\section{Preliminaries}

We begin by introducing the notion of the $\,^n \Delta_k$ rotation and associated sets.

Let $S \subseteq \mathbb{R}^n$ be a set of points in $n$-dimensional Euclidean space. The $\,^n \Delta_k$ configuration is defined as a system of $k$ lines in $\mathbb{R}^n$ that are mutually equiangular and intersect at a single point $P(n, k)$, referred to as the \textbf{configuration point}.

Corresponding to this configuration, we define the $\,^n \Delta_k$ sets as a collection of $k+1$ independent sets:  
\begin{itemize}
    \item $k$ of these are termed \textit{parametric sets}, each representing the lengths assigned to the respective $k$ lines of the configuration.  
    \item The remaining set is called the \textit{angular set}, which governs the direction or orientation of the entire configuration in space.
\end{itemize}

At each instance of rotation determined by an element of the angular set, the $k$ lines act as position vectors with respective magnitudes from the parametric sets. Together, they describe a subset of points in $\mathbb{R}^n$ relative to the configuration point $P(n, k)$.

As we rotate the configuration, at the point $P(n, k)$ by the angles in the angular set in order, we refer to this as the \emph{Delta Rotation}.

In this paper, we primarily consider the specific configuration $\,^2 \Delta_4$, representing two mutually perpendicular lines in the plane intersecting at a point $P(2, 4)$, which we shall simply denote by $P$. For ease of reference, we denote the configuration $\,^2 \Delta_4$ by $\Delta$ throughout the paper.

Now, we formally state the central conjecture under consideration:

\begin{conjecture}[Square Peg Problem]
Every simple, closed, non-self-intersecting Jordan curve $C$ inscribes a square; that is, there exists a square whose four vertices all lie on the boundary of $C$.
\end{conjecture}

\textbf{Conjecture 1}, widely known as the \emph{Square Peg Problem}, was first posed by \emph{Otto Toeplitz} in 1911 and remains one of the most intriguing open problems in elementary geometry.

\section{On $\Delta$ Rotations}

Throughout this section, we assume that $C$ denotes an arbitrary, closed, simple, and non-self-intersecting Jordan curve in the Euclidean plane $\mathbb{R}^2$. Let $P$ be a point strictly contained in the interior of the region enclosed by $C$, and let $P$ be referred to as the \emph{configuration point}. We define a \emph{$\Delta$-configuration} at $P$ as a pair of mutually perpendicular lines intersecting at $P$, extending in all four directions, such that each of the four resulting rays intersects the boundary of $C$.

\begin{proposition}
Let $\theta \in [0, 2\pi)$ be any arbitrary angle representing the orientation of a $\Delta$-configuration (i.e., rotation of the perpendicular pair about $P$). Then, for each such $\theta$, all four rays of the configuration touch the boundary of $C$ in finite, non-zero lengths. That is, none of the four segments connecting $P$ to $C$ along the rays have zero measure.
\end{proposition}

\begin{proof}
Since $P$ lies strictly inside the Jordan curve $C$, and $C$ is a simple, closed curve, it follows from the Jordan curve theorem that any ray emanating from $P$ must intersect the boundary of $C$ exactly once before escaping the interior. Given a configuration consisting of two perpendicular lines passing through $P$, these define four rays in distinct directions. Each of these rays, by the simplicity and closedness of $C$, must intersect the boundary at a unique point. Consequently, for each angle $\theta$, the segments of the rays from $P$ to their respective points of intersection with $C$ are finite and non-zero in length. Hence, none of the four directional segments in the $\Delta$-configuration can have zero measure. And hence, must have a possible measure so that it touches the boundary of $C$.
\end{proof}

By Proposition 1, we have established that for every orientation $\theta$, all four segments in the $\Delta$-configuration are non-degenerate. We now turn to the question of \emph{continuity} of these segment lengths as a function of rotation. Specifically, we ask: as $\theta$ varies continuously, do the corresponding segment lengths also vary continuously?

\begin{proposition}
Let $C$ and $P$ be as defined above. Then the mapping from orientation angle $\theta \in [0, 2\pi)$ to the vector of four segment lengths in the corresponding $\Delta$-configuration is continuous. That is, small changes in orientation produce small changes in segment lengths.
\end{proposition}

\begin{proof}
Let $T_\theta$ denote the $\Delta$-configuration obtained by orienting the pair of perpendicular lines through $P$ at angle $\theta$. Suppose for the sake of contradiction that the mapping $\theta \mapsto T_\theta$ is not continuous. Then there exists some $\theta_0$ and an $\varepsilon > 0$ such that for every $\delta > 0$, there exists $\theta'$ with $|\theta' - \theta_0| < \delta$, yet at least one of the segment lengths in $T_{\theta'}$ differs from that in $T_{\theta_0}$ by more than $\varepsilon$.

However, each segment length in $T_\theta$ is determined by the distance from $P$ to the intersection point of a ray (in direction determined by $\theta$) with the boundary of $C$. Since $C$ is a compact, continuous curve and the ray varies continuously with $\theta$, the intersection point—and hence the length—varies continuously with $\theta$. Thus, the segment lengths form continuous functions of $\theta$.

Therefore, our assumption of discontinuity leads to a contradiction, and we conclude that the segment lengths of the $\Delta$-configuration vary continuously as the orientation $\theta$ varies. This establishes the desired result.
\end{proof}

Building upon the propositions established in the previous section, we now pose a fundamental question concerning the transition between distinct $\Delta$-configurations within a fixed Jordan curve.

\begin{question}
Given two distinct $\Delta$-configurations at the same configuration point $P$ within a Jordan curve $C$, does there exist a connected angular subset $\Theta \subseteq [0, \pi)$ and corresponding parametric subsets such that a continuous rotation through $\Theta$ induces a continuous transition between the two configurations?
\end{question}

This question encapsulates the essence of rotational continuity within $\Delta$-configurations. Specifically, we are interested in determining whether it is always possible to rotate a given configuration smoothly to another—most notably, to its \emph{swapped} configuration, in which the original horizontal and vertical lines exchange roles.

If such a continuous transition exists, it implies the existence of a specific angle of rotation, denoted $\theta^*$, at which the configuration is symmetric with respect to the swap—that is, the lengths of the horizontal and vertical segments (measured from $P$ to their respective intersections with the boundary of $C$) are equal.

Let us define a real-valued function $f(\theta)$ representing the signed difference between the total lengths of the horizontal and vertical segments in the $\Delta$-configuration oriented at angle $\theta$. That is,
\[
f(\theta) := \ell_H(\theta) - \ell_V(\theta),
\]
where $\ell_H(\theta)$ and $\ell_V(\theta)$ denote the total lengths of the horizontal and vertical segments, respectively, in the configuration at orientation $\theta$.

From the continuity of the segment lengths established earlier, it follows that both $\ell_H(\theta)$ and $\ell_V(\theta)$ are continuous functions of $\theta$. Therefore, $f(\theta)$ is also continuous on $[0, \pi)$.

Suppose the initial configuration corresponds to an angle $\theta_0$ with $f(\theta_0) > 0$, and the swapped configuration corresponds to an angle $\theta_1$ with $f(\theta_1) < 0$. Then, by the Intermediate Value Theorem, there exists some $\theta^* \in (\theta_0, \theta_1)$ such that $f(\theta^*) = 0$.

This implies that at orientation $\theta^*$, the horizontal and vertical segments are of equal length. Due to the orthogonality of the lines in a $\Delta$-configuration, this equality implies that the configuration forms the diagonals of a square inscribed within the curve $C$ and centered at $P$. And hence proves the \emph{Conjecture 1}.