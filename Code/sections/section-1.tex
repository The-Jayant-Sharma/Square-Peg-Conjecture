\section{Preliminaries}

We begin by introducing the notion of the $\,^n \Delta_k$ rotation and associated sets.

Let $S \subseteq \mathbb{R}^n$ be a set of points in $n$-dimensional Euclidean space. The $\,^n \Delta_k$ configuration is defined as a system of $k$ lines in $\mathbb{R}^n$ that are mutually equiangular and intersect at a single point $P(n, k)$, referred to as the \textbf{configuration point}.

Corresponding to this configuration, we define the $\,^n \Delta_k$ sets as a collection of $k+1$ independent sets:  
\begin{itemize}
    \item $k$ of these are termed \textit{parametric sets}, each representing the lengths assigned to the respective $k$ lines of the configuration.  
    \item The remaining set is called the \textit{angular set}, which governs the direction or orientation of the entire configuration in space.
\end{itemize}

At each instance of rotation determined by an element of the angular set, the $k$ lines act as position vectors with respective magnitudes from the parametric sets. Together, they describe a subset of points in $\mathbb{R}^n$ relative to the configuration point $P(n, k)$.

In this paper, we primarily consider the specific configuration $\,^2 \Delta_4$, representing two mutually perpendicular lines in the plane intersecting at a point $P(2, 4)$, which we shall simply denote by $P$. For ease of reference, we denote the configuration $\,^2 \Delta_4$ by $\Delta$ throughout the paper.

Now, we formally state the central conjecture under consideration:

\begin{conjecture}[Square Peg Problem]
Every simple, closed, non-self-intersecting Jordan curve $C$ inscribes a square; that is, there exists a square whose four vertices all lie on the boundary of $C$.
\end{conjecture}

\textbf{Conjecture 1}, widely known as the \emph{Square Peg Problem}, was first posed by \emph{Otto Toeplitz} in 1911 and remains one of the most intriguing open problems in elementary geometry.
