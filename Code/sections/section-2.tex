\section{On the Approach}

We now turn our attention to \emph{Conjecture 1}. To investigate it, let us consider an arbitrary Jordan curve $C$, and a point $P$ located in its interior. We construct two perpendicular lines passing through $P$, each intersecting the boundary of $C$. This initial setup is referred to as the \emph{initial configuration}, denoted by $c_0$. By construction, both lines intersect at the point $P$.

As previously defined, we introduce the notion of a $\Delta$-rotation of $c_0$, which refers to the simultaneous motion of the two lines along the boundary of $C$, while preserving both their orthogonality and intersection at $P$. As the configuration evolves through this $\Delta$-rotation, each new position is called a \emph{successor configuration}, while prior positions are \emph{predecessor configurations}.

We now introduce a key concept, called a \emph{restriction}. Suppose that after a certain amount of $\Delta$-rotation, the configuration reaches a position in which it is no longer possible to continue rotating in the initial angular direction. In this case, we refer to this situation as a \emph{restriction} in the motion of the configuration.

\begin{proposition}
A restriction can occur only if at least one of the four endpoints of the configuration lies tangent to a convex region of the curve $C$; specifically, at a local extremum of a locally upward (or downward) region.
\end{proposition}

\begin{proof}
By the definition of a restriction, suppose that the configuration reaches a position $c_1$ and cannot proceed further in the direction of angular increase, say $\theta$. Let this configuration be at angular coordinate $\theta_f$. By continuity (as established in Proposition 2), the existence of the restriction implies that at least one of the four endpoints of $c_1$ has no corresponding continuation at angle $\theta_f + d\theta$ for any infinitesimal increment $d\theta > 0$.

However, since the curve $C$ is continuous, a neighboring point must exist, and it must lie on the opposite side, at angle $\theta_f - d\theta$. Due to the closure of $C$, the configuration at this neighboring angular position cannot coincide with the initial configuration $c_0$, and hence must itself possess both a valid successor and predecessor.

This behavior characterizes a situation in which one endpoint of $c_1$ is tangent to a locally convex region of the curve. In such a case, the configuration reaches a local extremum—where the tangent line cannot move further in the current angular direction without violating the orthogonality or containment conditions. Thus, the predecessor configuration exists at $\theta_f - d\theta$, and any valid continuation must emerge on the opposite side.

\end{proof}

Hence, by Proposition 3, we conclude that restrictions in the $\Delta$-rotation process can occur only at tangential points lying on convex regions of $C$, relative to the center of configuration $P$. Moreover, even at such points, a successor configuration must still exist, as dictated by the nature of convex regions in continuous curves: a local extremum must, by continuity, be flanked by regions with both positive and negative slope. Therefore, a configuration arriving from the side of a positive slope must be able to continue onto the side of a negative slope, or vice versa.

Hence for the sake of our proof, we propose another preposition:

\begin{proposition}
For curve $C$ having the properties as described above, if initiated by an initial configuration $_0$ such that $c_0$ has a predecessor as well as a successor configuration. Then as we $\Delta$-Rotate the configuration $c_0$ it everytime may produce an unique configuration.
\end{proposition}
\begin{proof}
For some initial configuration $c_0$ which has a successor as well as a predecessor configuration. Now we know that, every final configuration $c_i$ afterwards has a predecessor configuration for certain.
Now we assume that for some final configuration $c_n$ there is a 
\end{proof}
