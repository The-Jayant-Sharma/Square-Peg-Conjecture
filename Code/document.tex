\documentclass[9pt]{amsart}
\usepackage[utf8]{inputenc}
\usepackage{amsmath, amssymb, amsthm}
\usepackage{mathrsfs, graphicx, tikz}
\usepackage[left=3cm, right=3cm, bottom=3.4cm]{geometry}
\usepackage{hyperref}
\usepackage{fancyhdr}

\renewcommand{\qedsymbol}{\emph{(end of proof)}}

\theoremstyle{plain}
\newtheorem{theorem}{Theorem}
\renewcommand{\thetheorem}{\Roman{theorem}}

\newtheorem{definition}{Definition}
\renewcommand{\thedefinition}{\Roman{definition}}

\newtheorem{lemma}{Lemma}
\renewcommand{\thelemma}{\Roman{lemma}}

\newtheorem{conjecture}{Conjecture}
\renewcommand{\theconjecture}{\Roman{conjecture}}


\title{\textbf{On the Square Peg Problem}}
\author{Jayant Sharma}
\date{\today}

\pagestyle{fancy}
\fancyhf{}
\fancyhead[L]{\emph{On the Square Peg Problem, Jayant Sharma}}
\fancyhead[R]{\thepage}
\renewcommand{\headrulewidth}{0.4pt}
\renewcommand{\footrulewidth}{0pt}

\begin{document}
\maketitle
\begin{abstract}
The \emph{Square Peg Problem}, introduced by \emph{Otto Toeplitz} in 1911, asks whether every simple closed Jordan curve $C$ inscribes a square—that is, whether four points on $C$ can form the vertices of a square. To approach this, we introduce the notion of \emph{$4\text{-}\Delta$} sets and rotations, as a special case of the more general $^n\Delta_k$ configurations, defined as sets of $k$ lines in $n$-dimensional space with equal mutual angles.

For $n = 2$ and $k = 4$, this configuration consists of two perpendicular line pairs (angle $\pi/2$). We prove that the continuous rotation of such a configuration over a Jordan curve causes the vertical and horizontal line pairs to swap positions. By the \emph{Intermediate Value Theorem}, this implies the existence of a moment when vertical and horizontal lines coincide in length, thus forming the diagonals of an inscribed square.
\end{abstract}
\tableofcontents

\nocite{*}

\bibliographystyle{plain}
\bibliography{document}

\end{document}
